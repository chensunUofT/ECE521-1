%%%%%%%%%%%%%%%%%%%%%%%%%%%%%%%%%%%%%%%%%
% Programming/Coding Assignment
% LaTeX Template
%
% This template has been downloaded from:
% http://www.latextemplates.com
%
% Original author:
% Ted Pavlic (http://www.tedpavlic.com)
%
% Note:
% The \lipsum[#] commands throughout this template generate dummy text
% to fill the template out. These commands should all be removed when 
% writing assignment content.
%
% This template uses a Perl script as an example snippet of code, most other
% languages are also usable. Configure them in the "CODE INCLUSION 
% CONFIGURATION" section.
%
%%%%%%%%%%%%%%%%%%%%%%%%%%%%%%%%%%%%%%%%%

%----------------------------------------------------------------------------------------
%	PACKAGES AND OTHER DOCUMENT CONFIGURATIONS
%----------------------------------------------------------------------------------------

\documentclass{article}

\usepackage{fancyhdr} % Required for custom headers
\usepackage{lastpage} % Required to determine the last page for the footer
\usepackage{extramarks} % Required for headers and footers
\usepackage[usenames,dvipsnames]{color} % Required for custom colors
\usepackage{graphicx} % Required to insert images
\usepackage{listings} % Required for insertion of code
\usepackage{courier} % Required for the courier font
\usepackage{lipsum} % Used for inserting dummy 'Lorem ipsum' text into the template
\usepackage{pgfplotstable} % Import csv data
\usepackage{mathtools}
\usepackage{amsmath,amssymb} % Math equation tools
\usepackage{hyperref} % Create references\pgfplotsset{compat=1.12}
\usepackage{filecontents}% Used so that the external files can be placed in this file
\usepackage{pgf} % Loops in latex
\usepackage{bookmark}
\usepackage{booktabs}
\usepackage{float}
\usepackage{siunitx} % Formats the units and values
\usepackage{parskip}
\usepackage{chngcntr}
\usepackage{tikz}
\usetikzlibrary{bayesnet}

\setlength{\parskip}{10pt} % 1ex plus 0.5ex minus 0.2ex}

% Setup siunitx:
\sisetup{
  round-mode          = places, % Rounds numbers
  round-precision     = 4, % to 2 places
}

% Margins
\topmargin=-0.45in
\evensidemargin=0in
\oddsidemargin=0in
\textwidth=6.5in
\textheight=9.0in
\headsep=0.25in

\linespread{1.1} % Line spacing
\renewcommand{\tableautorefname}{Table} % PS
\renewcommand{\equationautorefname}{Eq.} % PS

% Set up the header and footer
\pagestyle{fancy}
\chead{\hmwkClass\ (\hmwkClassTime): \hmwkTitle} % Top center head
\rhead{\firstxmark} % Top right header
\lhead{}
\lfoot{\lastxmark} % Bottom left footer
\cfoot{} % Bottom center footer
\rfoot{Page\ \thepage\ of\ \protect\pageref{LastPage}} % Bottom right footer
\renewcommand\headrulewidth{0.4pt} % Size of the header rule
\renewcommand\footrulewidth{0.4pt} % Size of the footer rule

\renewcommand{\lstlistingname}{Code}
\newcommand{\indep}{\mathrel{\text{\scalebox{1.07}{$\perp\mkern-10mu\perp$}}}}

\DeclareGraphicsExtensions{.eps,.pdf,.png,.jpg}

\setlength\parindent{0pt} % Removes all indentation from paragraphs

%----------------------------------------------------------------------------------------
%	CODE INCLUSION CONFIGURATION
%----------------------------------------------------------------------------------------

\definecolor{MyDarkGreen}{rgb}{0.0,0.4,0.0} % This is the color used for comments
\lstloadlanguages{Perl} % Load Perl syntax for listings, for a list of other languages supported see: ftp://ftp.tex.ac.uk/tex-archive/macros/latex/contrib/listings/listings.pdf
\lstset{language=Perl, % Use Perl in this example
        frame=single, % Single frame around code
        basicstyle=\small\ttfamily, % Use small true type font
        keywordstyle=[1]\color{Blue}\bf, % Perl functions bold and blue
        keywordstyle=[2]\color{Purple}, % Perl function arguments purple
        keywordstyle=[3]\color{Blue}\underbar, % Custom functions underlined and blue
        identifierstyle=, % Nothing special about identifiers                                         
        commentstyle=\usefont{T1}{pcr}{m}{sl}\color{MyDarkGreen}\small, % Comments small dark green courier font
        stringstyle=\color{Purple}, % Strings are purple
        showstringspaces=false, % Don't put marks in string spaces
        tabsize=5, % 5 spaces per tab
        %
        % Put standard Perl functions not included in the default language here
        morekeywords={rand},
        %
        % Put Perl function parameters here
        morekeywords=[2]{on, off, interp},
        %
        % Put user defined functions here
        morekeywords=[3]{test},
       	%
        morecomment=[l][\color{Blue}]{...}, % Line continuation (...) like blue comment
        numbers=left, % Line numbers on left
        firstnumber=1, % Line numbers start with line 1
        numberstyle=\tiny\color{Blue}, % Line numbers are blue and small
        stepnumber=5 % Line numbers go in steps of 5
}

% Creates a new command to include a perl script, the first parameter is the filename of the script (without .pl), the second parameter is the caption
\newcommand{\perlscript}[2]{
\begin{itemize}
\item[]\lstinputlisting[caption=#2,label=#1]{#1.pl}
\end{itemize}
}

%----------------------------------------------------------------------------------------
%	DOCUMENT STRUCTURE COMMANDS
%	Skip this unless you know what you're doing
%----------------------------------------------------------------------------------------

% Header and footer for when a page split occurs within a problem environment
\newcommand{\enterProblemHeader}[1]{
%\nobreak\extramarks{#1}{#1 continued on next page\ldots}\nobreak
%\nobreak\extramarks{#1 (continued)}{#1 continued on next page\ldots}\nobreak
}

% Header and footer for when a page split occurs between problem environments
\newcommand{\exitProblemHeader}[1]{
%\nobreak\extramarks{#1 (continued)}{#1 continued on next page\ldots}\nobreak
%\nobreak\extramarks{#1}{}\nobreak
}

\setcounter{secnumdepth}{0} % Removes default section numbers
\newcounter{homeworkProblemCounter} % Creates a counter to keep track of the number of problems

\newcommand{\homeworkProblemName}{}
\newenvironment{homeworkProblem}[1][Task \arabic{homeworkProblemCounter}]{ % Makes a new environment called homeworkProblem which takes 1 argument (custom name) but the default is "Problem #"
\def\sectionautorefname{Task}
\refstepcounter{homeworkProblemCounter}%
\renewcommand{\homeworkProblemName}{#1} % Assign \homeworkProblemName the name of the problem
\section{\homeworkProblemName} % Make a section in the document with the custom problem count
\enterProblemHeader{\homeworkProblemName} % Header and footer within the environment
}{
\exitProblemHeader{\homeworkProblemName} % Header and footer after the environment
}

\newcommand{\problemAnswer}[1]{ % Defines the problem answer command with the content as the only argument
\noindent\framebox[\columnwidth][c]{\begin{minipage}{0.98\columnwidth}#1\end{minipage}} % Makes the box around the problem answer and puts the content inside
}

\newcommand{\homeworkSectionName}{}
\newenvironment{homeworkSection}[1]{ % New environment for sections within homework problems, takes 1 argument - the name of the section
\renewcommand{\homeworkSectionName}{#1} % Assign \homeworkSectionName to the name of the section from the environment argument
\subsection{\homeworkSectionName} % Make a subsection with the custom name of the subsection
\enterProblemHeader{\homeworkProblemName\ [\homeworkSectionName]} % Header and footer within the environment
}{
\enterProblemHeader{\homeworkProblemName} % Header and footer after the environment
}

%----------------------------------------------------------------------------------------
%	NAME AND CLASS SECTION
%----------------------------------------------------------------------------------------

\newcommand{\hmwkTitle}{Assignment 4} % Assignment title
\newcommand{\hmwkDueDate}{Thursday,\ April\ 7,\ 2016} % Due date
\newcommand{\hmwkClass}{ECE521} % Course/class
\newcommand{\hmwkClassTime}{L0101} % Class/lecture time
\newcommand{\hmwkAuthorName}{Davi Frossard} % Your name

%----------------------------------------------------------------------------------------
%	TITLE PAGE
%----------------------------------------------------------------------------------------

\title{
\vspace{2in}
\textmd{\textbf{\hmwkClass:\ \hmwkTitle}}\\
\normalsize\vspace{0.1in}\small{Due\ on\ \hmwkDueDate}\\
\vspace{0.1in}
\vspace{3in}
}

\author{\textbf{Davi Frossard} \\ \textbf{Erich Sato} \\ \textbf{Mauro Brito}}
%\date{} % Insert date here if you want it to appear below your name

%----------------------------------------------------------------------------------------

\begin{document}

\maketitle
\clearpage


%----------------------------------------------------------------------------------------
%	PART 1
%----------------------------------------------------------------------------------------

\begin{center}
\section{Graphical Models}
\homeworkSectionName{Graphical models from factorization}
\end{center}

\begin{homeworkProblem}
\label{gm1}

If we draw a Bayesian network representation of \autoref{eq1} we get \autoref{bn1}.
\begin{equation}
\label{eq1}
P(a,b,c,d,e,f) = P(a|b)P(b)P(c|a,b)P(d|b)P(e|c)P(f|b,e)
\end{equation}

\begin{figure}[ht]
	\begin{center}
		\tikz{ %
			\node[latent] (e) {$e$} ; %
			\node[obs, below=of e, xshift=-1cm] (f) {$f$} ; %
			\node[latent, above=of e, xshift=-2cm] (d) {$d$} ; %
			\node[latent, above=of e] (c) {$c$} ; %
			\node[latent, above=of c, xshift=1cm] (a) {$a$} ; %
			\node[latent, above=of a, xshift=-2cm] (b) {$b$} ; %
			\edge {e,b} {f} ; %
			\edge {c} {e} ; %
			\edge {b,a} {c} ; %
			\edge {b} {a} ; %
			\edge {b} {d} ; %
		}
	\end{center}
	\caption{Bayesian Network Representation of \autoref{eq1}.}
	\label{bn1}
\end{figure}


\clearpage
\end{homeworkProblem}
%----------------------------------------------------------------------------------------

%----------------------------------------------------------------------------------------
%	PART 2
%----------------------------------------------------------------------------------------

\begin{homeworkProblem}
\label{gm2}

Let us rewrite the joint distribution from \autoref{eq1} as \autoref{eq2}.
\begin{equation}
\label{eq2}
P(a,b,c,d,e,f) = f_1(a,b)f_2(b)f_3(c,a,b)f_4(d,b)f_5(e,c)f_6(f,b,e)
\end{equation}	

If we now draw the corresponding factor graph we obtain \autoref{fg1}.

\begin{figure}[ht]
	\begin{center}
		\tikz{ %
			\node[latent] (b) {$b$};
			\node[latent, below=of b, xshift=2cm] (a) {$a$};
			\node[latent, below=of a] (c) {$c$};
			\node[latent, left=of c, xshift=-3cm] (d) {$d$};
			\node[latent, below=of c, xshift=2cm] (e) {$e$};
			\node[obs, below=of e, xshift=-4cm] (f) {$f$};
			\factor[right=of b] {f1} {$f_1$} {b} {a};
			\factor[above=of b] {f2} {$f_2$} {} {b};
			\factor[left=of a] {f3} {$f_3$} {a,b} {c};
			\factor[left=of b] {f4} {$f_4$} {b} {d};
			\factor[right=of c] {f5} {$f_5$} {c} {e};
			\factor[left=of e, xshift=-3.15cm]{f6} {180:$f_6$} {b,e} {f};
		}
	\end{center}
	\caption{Factor Graph Representation of \autoref{eq2}.}
	\label{fg1}
\end{figure}

\clearpage
\end{homeworkProblem}
%----------------------------------------------------------------------------------------


%----------------------------------------------------------------------------------------
%	PART 3
%----------------------------------------------------------------------------------------

\begin{homeworkProblem}
\label{gm3}

%TODO

\clearpage
\end{homeworkProblem}
%----------------------------------------------------------------------------------------

%----------------------------------------------------------------------------------------
%	PART 4
%----------------------------------------------------------------------------------------

\begin{center}
	\subsection{Conversion between graphical models}
	\setcounter{homeworkProblemCounter}{0}
\end{center}

\begin{homeworkProblem}
\label{gm4}

%TODO

\clearpage
\end{homeworkProblem}
%----------------------------------------------------------------------------------------

%----------------------------------------------------------------------------------------
%	PART 5
%----------------------------------------------------------------------------------------

\begin{homeworkProblem}
\label{gm5}

%TODO

\clearpage
\end{homeworkProblem}
%----------------------------------------------------------------------------------------

%----------------------------------------------------------------------------------------
%	PART 6
%----------------------------------------------------------------------------------------

\begin{homeworkProblem}
\label{gm6}

%TODO

\clearpage
\end{homeworkProblem}
%----------------------------------------------------------------------------------------

%----------------------------------------------------------------------------------------
%	PART 7
%----------------------------------------------------------------------------------------

\begin{center}
	\subsection{Conditional Independence in Bayesian Networks}
	\setcounter{homeworkProblemCounter}{0}
\end{center}

\begin{homeworkProblem}
\label{gm7}

We can express the joint probability of the Bayesian Network provided using \autoref{eq3}.

\begin{equation}
\label{eq3}
P(a,b,c,d,e,f) = P(a)P(b|a)P(f|b)P(e)P(c|b,e)P(d|b,c)
\end{equation}

\clearpage
\end{homeworkProblem}
%----------------------------------------------------------------------------------------

%----------------------------------------------------------------------------------------
%	PART 8
%----------------------------------------------------------------------------------------

\begin{homeworkProblem}
\label{gm8}

\begin{itemize}
\item $\boldsymbol{a\indep c}$ : FALSE 
\subitem Consider $P(c) = \begin{dcases} 1 & P(b) + P(e) > 1\\0 & otherwise\end{dcases}$, now let us assume $P(b) = P(a) = \mathcal{N}(0,1)$. Therefore we have  $P(c) = \begin{dcases} 1 & P(a) + P(e) > 1\\0 & otherwise\end{dcases}$, which clearly shows that \textbf{a} is not independent of \textbf{c}.

\item $\boldsymbol{a\indep c|b}$ : TRUE 
\subitem If we marginalize the distribution in \autoref{gm7} over \textbf{a} and \textbf{c} we obtain $P(a,c|b) = P(a)P(b|a)P(e)P(c|b,e)$

\end{itemize}

\clearpage
\end{homeworkProblem}
%----------------------------------------------------------------------------------------

\end{document}
