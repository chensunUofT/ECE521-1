%%%%%%%%%%%%%%%%%%%%%%%%%%%%%%%%%%%%%%%%%
% Programming/Coding Assignment
% LaTeX Template
%
% This template has been downloaded from:
% http://www.latextemplates.com
%
% Original author:
% Ted Pavlic (http://www.tedpavlic.com)
%
% Note:
% The \lipsum[#] commands throughout this template generate dummy text
% to fill the template out. These commands should all be removed when 
% writing assignment content.
%
% This template uses a Perl script as an example snippet of code, most other
% languages are also usable. Configure them in the "CODE INCLUSION 
% CONFIGURATION" section.
%
%%%%%%%%%%%%%%%%%%%%%%%%%%%%%%%%%%%%%%%%%

%----------------------------------------------------------------------------------------
%	PACKAGES AND OTHER DOCUMENT CONFIGURATIONS
%----------------------------------------------------------------------------------------

\documentclass{article}

\usepackage{fancyhdr} % Required for custom headers
\usepackage{lastpage} % Required to determine the last page for the footer
\usepackage{extramarks} % Required for headers and footers
\usepackage[usenames,dvipsnames]{color} % Required for custom colors
\usepackage{graphicx} % Required to insert images
\usepackage{subcaption}
\usepackage{listings} % Required for insertion of code
\usepackage{courier} % Required for the courier font
\usepackage{lipsum} % Used for inserting dummy 'Lorem ipsum' text into the template
\usepackage{pgfplotstable} % Import csv data
\usepackage{amsmath} % Math equation tools
\usepackage[pdftex]{hyperref} % Create references

% Margins
\topmargin=-0.45in
\evensidemargin=0in
\oddsidemargin=0in
\textwidth=6.5in
\textheight=9.0in
\headsep=0.25in

\linespread{1.1} % Line spacing

% Set up the header and footer
\pagestyle{fancy}
\lhead{\hmwkAuthorName} % Top left header
\chead{\hmwkClass\ (\hmwkClassTime): \hmwkTitle} % Top center head
%\rhead{\firstxmark} % Top right header
\lfoot{\lastxmark} % Bottom left footer
\cfoot{} % Bottom center footer
\rfoot{Page\ \thepage\ of\ \protect\pageref{LastPage}} % Bottom right footer
\renewcommand\headrulewidth{0.4pt} % Size of the header rule
\renewcommand\footrulewidth{0.4pt} % Size of the footer rule

\setlength\parindent{0pt} % Removes all indentation from paragraphs

%----------------------------------------------------------------------------------------
%	CODE INCLUSION CONFIGURATION
%----------------------------------------------------------------------------------------

\definecolor{MyDarkGreen}{rgb}{0.0,0.4,0.0} % This is the color used for comments
\lstloadlanguages{Perl} % Load Perl syntax for listings, for a list of other languages supported see: ftp://ftp.tex.ac.uk/tex-archive/macros/latex/contrib/listings/listings.pdf
\lstset{language=Perl, % Use Perl in this example
        frame=single, % Single frame around code
        basicstyle=\small\ttfamily, % Use small true type font
        keywordstyle=[1]\color{Blue}\bf, % Perl functions bold and blue
        keywordstyle=[2]\color{Purple}, % Perl function arguments purple
        keywordstyle=[3]\color{Blue}\underbar, % Custom functions underlined and blue
        identifierstyle=, % Nothing special about identifiers                                         
        commentstyle=\usefont{T1}{pcr}{m}{sl}\color{MyDarkGreen}\small, % Comments small dark green courier font
        stringstyle=\color{Purple}, % Strings are purple
        showstringspaces=false, % Don't put marks in string spaces
        tabsize=5, % 5 spaces per tab
        %
        % Put standard Perl functions not included in the default language here
        morekeywords={rand},
        %
        % Put Perl function parameters here
        morekeywords=[2]{on, off, interp},
        %
        % Put user defined functions here
        morekeywords=[3]{test},
       	%
        morecomment=[l][\color{Blue}]{...}, % Line continuation (...) like blue comment
        numbers=left, % Line numbers on left
        firstnumber=1, % Line numbers start with line 1
        numberstyle=\tiny\color{Blue}, % Line numbers are blue and small
        stepnumber=5 % Line numbers go in steps of 5
}

% Creates a new command to include a perl script, the first parameter is the filename of the script (without .pl), the second parameter is the caption
\newcommand{\perlscript}[2]{
\begin{itemize}
\item[]\lstinputlisting[caption=#2,label=#1]{#1.pl}
\end{itemize}
}

%----------------------------------------------------------------------------------------
%	DOCUMENT STRUCTURE COMMANDS
%	Skip this unless you know what you're doing
%----------------------------------------------------------------------------------------

% Header and footer for when a page split occurs within a problem environment
\newcommand{\enterProblemHeader}[1]{
%\nobreak\extramarks{#1}{#1 continued on next page\ldots}\nobreak
%\nobreak\extramarks{#1 (continued)}{#1 continued on next page\ldots}\nobreak
}

% Header and footer for when a page split occurs between problem environments
\newcommand{\exitProblemHeader}[1]{
%\nobreak\extramarks{#1 (continued)}{#1 continued on next page\ldots}\nobreak
%\nobreak\extramarks{#1}{}\nobreak
}

\setcounter{secnumdepth}{0} % Removes default section numbers
\newcounter{homeworkProblemCounter} % Creates a counter to keep track of the number of problems
\setcounter{homeworkProblemCounter}{0}

\newcommand{\homeworkProblemName}{}
\newenvironment{homeworkProblem}[1][Task \arabic{homeworkProblemCounter}]{ % Makes a new environment called homeworkProblem which takes 1 argument (custom name) but the default is "Problem #"
\def\sectionautorefname{Task}
\refstepcounter{homeworkProblemCounter}%
\renewcommand{\homeworkProblemName}{#1} % Assign \homeworkProblemName the name of the problem
\section{\homeworkProblemName} % Make a section in the document with the custom problem count
\enterProblemHeader{\homeworkProblemName} % Header and footer within the environment
}{
\exitProblemHeader{\homeworkProblemName} % Header and footer after the environment
}

\newcommand{\problemAnswer}[1]{ % Defines the problem answer command with the content as the only argument
\noindent\framebox[\columnwidth][c]{\begin{minipage}{0.98\columnwidth}#1\end{minipage}} % Makes the box around the problem answer and puts the content inside
}

\newcommand{\homeworkSectionName}{}
\newenvironment{homeworkSection}[1]{ % New environment for sections within homework problems, takes 1 argument - the name of the section
\renewcommand{\homeworkSectionName}{#1} % Assign \homeworkSectionName to the name of the section from the environment argument
\subsection{\homeworkSectionName} % Make a subsection with the custom name of the subsection
\enterProblemHeader{\homeworkProblemName\ [\homeworkSectionName]} % Header and footer within the environment
}{
\enterProblemHeader{\homeworkProblemName} % Header and footer after the environment
}

%----------------------------------------------------------------------------------------
%	NAME AND CLASS SECTION
%----------------------------------------------------------------------------------------

\newcommand{\hmwkTitle}{Assignment 1} % Assignment title
\newcommand{\hmwkDueDate}{Monday,\ January\ 25,\ 2016} % Due date
\newcommand{\hmwkClass}{ECE521} % Course/class
\newcommand{\hmwkClassTime}{L0101} % Class/lecture time
\newcommand{\hmwkAuthorName}{Davi Frossard} % Your name

%----------------------------------------------------------------------------------------
%	TITLE PAGE
%----------------------------------------------------------------------------------------

\title{
\vspace{2in}
\textmd{\textbf{\hmwkClass:\ \hmwkTitle}}\\
\normalsize\vspace{0.1in}\small{Due\ on\ \hmwkDueDate}\\
\vspace{0.1in}
\vspace{3in}
}

\author{\textbf{\hmwkAuthorName}}
%\date{} % Insert date here if you want it to appear below your name

%----------------------------------------------------------------------------------------

\begin{document}

\maketitle
\clearpage


%----------------------------------------------------------------------------------------
%	PROBLEM 1
%----------------------------------------------------------------------------------------


\begin{homeworkProblem}
\label{t:1}

\begin{center}
\pgfplotstabletypeset[
    col sep=space,
    string type,
    display columns/0/.style={column name=\textbf{Training Set Size (N)}, column type={|c|}},
    display columns/1/.style={column name=\textbf{Validation Errors}, column type={c|}},
    every head row/.style={before row=\hline,after row=\hline},
    every last row/.style={after row=\hline},
    ]{results/Task_1.csv}
\end{center}

Using a larger training set increased the algorithm's accuracy since it makes data more diverse and harder to over-fit.


\clearpage
\end{homeworkProblem}

%----------------------------------------------------------------------------------------



%----------------------------------------------------------------------------------------
%	PROBLEM 2
%----------------------------------------------------------------------------------------


\begin{homeworkProblem}

\begin{center}
\pgfplotstabletypeset[
    col sep=space,
    string type,
    display columns/0/.style={column name=\textbf{Neighbors Considered (K)}, column type={|c|}},
    display columns/1/.style={column name=\textbf{Validation Errors}, column type={c|}},
    every head row/.style={before row=\hline,after row=\hline},
    every last row/.style={after row=\hline},
    ]{results/Task_2.csv}
\end{center}

Smaller values of K tend to lead to over-fitting when the decision boundary for a variable is complex, that was not the case for this dataset, whose optimum K was 3; on the other hand an over-smoothed decision boundary (larger K) penalized accuracy.

\clearpage
\end{homeworkProblem}

%----------------------------------------------------------------------------------------



%----------------------------------------------------------------------------------------
%	PROBLEM 3
%----------------------------------------------------------------------------------------


\begin{homeworkProblem}

\begin{figure*}[h!]
    \centering
    \includegraphics[width=\columnwidth]{results/Task_3.eps}
    \caption{Linear regression of artificial dataset and cost plots}
    \label{fig:task3}
\end{figure*}

We first normalize the dataset by the maximum value to prevent overflows. Fitting the data to a first order polynomial with stochastic gradient descent yields the following model:

\begin{figure*}[h!]
    \centering
    \scalebox{1}{\input{results/Task_3_eq.pgf}}
    \label{fig:task4_eq}
\end{figure*}


\clearpage
\end{homeworkProblem}

%----------------------------------------------------------------------------------------



%----------------------------------------------------------------------------------------
%	PROBLEM 4
%----------------------------------------------------------------------------------------


\begin{homeworkProblem}

\begin{figure*}[h!]
    \centering
    \includegraphics[width=\columnwidth]{results/Task_4.eps}
    \caption{Multivariate regression of artificial dataset and cost plots}
    \label{fig:task4}
\end{figure*}

By using non linear parameters we give more malleability to the model since it can now make curves, which makes the final cost decrease considerably.

\begin{figure*}[h!]
    \centering
    \scalebox{0.85}{\input{results/Task_4_eq.pgf}}
    \label{fig:task4_eq}
\end{figure*}


\clearpage
\end{homeworkProblem}

%----------------------------------------------------------------------------------------



%----------------------------------------------------------------------------------------
%	PROBLEM 5
%----------------------------------------------------------------------------------------

\begin{homeworkProblem}

\begin{center}
\pgfplotstabletypeset[
    col sep=space,
    string type,
    display columns/0/.style={column name=\textbf{Training Set Size (N)}, column type={|c|}},
    display columns/1/.style={column name=\textbf{Validation Errors}, column type={c|}},
    every head row/.style={before row=\hline,after row=\hline},
    every last row/.style={after row=\hline},
    ]{results/Task_5.csv}
\end{center}


For the same reason as discussed in \autoref{t:1}, larger values of N provide better results since it makes the dataset less prone to over-fitting.

\clearpage
\end{homeworkProblem}

%----------------------------------------------------------------------------------------



%----------------------------------------------------------------------------------------
%	PROBLEM 6
%----------------------------------------------------------------------------------------

\begin{homeworkProblem}

\begin{figure*}[h!]
    \centering
    \includegraphics[width=\columnwidth]{results/Task_6.eps}
    \caption{Classification errors by epoch}
    \label{fig:task6}
\end{figure*}

\clearpage
\end{homeworkProblem}

%----------------------------------------------------------------------------------------



%----------------------------------------------------------------------------------------
%	PROBLEM 7
%----------------------------------------------------------------------------------------

\begin{homeworkProblem}

\begin{center}
\pgfplotstabletypeset[
    col sep=space,
    string type,
    display columns/0/.style={column name=$\boldsymbol{\lambda}$, column type={|c|}},
    display columns/1/.style={column name=\textbf{Validation Errors}, column type={c|}},
    every head row/.style={before row=\hline,after row=\hline},
    every last row/.style={after row=\hline},
    ]{results/Task_7.csv}
\end{center}

A value of $\lambda = 0.1$ improves the model by decaying the weights and reducing over-fitting, improving the accuracy even with such a small dataset (50 points).

\clearpage
\end{homeworkProblem}

%----------------------------------------------------------------------------------------

\end{document}